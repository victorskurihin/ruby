%% LyX 2.1.2 created this file.  For more info, see http://www.lyx.org/.
%% Do not edit unless you really know what you are doing.
\documentclass[russian]{article}
\usepackage[T2A,T1]{fontenc}
\usepackage[utf8]{inputenc}
\usepackage{geometry}
\geometry{verbose,tmargin=1cm,bmargin=1cm,lmargin=1cm,rmargin=1cm}
\usepackage{units}
\usepackage{amssymb}
\usepackage{graphicx}

\makeatletter

%%%%%%%%%%%%%%%%%%%%%%%%%%%%%% LyX specific LaTeX commands.
\DeclareRobustCommand{\cyrtext}{%
  \fontencoding{T2A}\selectfont\def\encodingdefault{T2A}}
\DeclareRobustCommand{\textcyr}[1]{\leavevmode{\cyrtext #1}}
\AtBeginDocument{\DeclareFontEncoding{T2A}{}{}}

%% A simple dot to overcome graphicx limitations
\newcommand{\lyxdot}{.}


\makeatother

\usepackage{babel}
\begin{document}

\title{Суммы, быстро растущие функции, и другие насущные вещи | Шаг 8 }

\maketitle
Найдите \textit{асимптотику} суммы $\sum_{k=1}^{n}\frac{k!}{2^{k}}$
при $n\rightarrow\infty$. В тексте своего решения чётко указывайте,
на какие куски Вы разбиваете сумму и как эти куски оцениваете. В Вашем
окончательном ответе не должно быть ни факториалов, ни сумм с растущим
количеством слагаемых.

\rule[0.5ex]{0.27\columnwidth}{1pt}

Комментарий от преподавателя:

Ответ: $\sum\sim\frac{n!}{2^{n}}\sim\sqrt{2\cdot\pi}\cdot\sqrt{n}\cdot\left(2e\right)^{-n}\cdot n^{n}$,
то есть последнее слагаемое определяет асимптотику суммы. Одно из
возможных разбиений на слагаемые: $1\leqslant k\leqslant n-2$, $k=n-1$,
$k=n$. В первом случае:
\[
\sum_{k=1}^{n}\frac{k!}{2^{k}}\leqslant n\cdot\frac{\left(n-1\right)!}{2^{n-2}}=\underline{O}\left(\frac{\left(n-1\right)!}{2^{n}}\right)=\bar{o}\left(\frac{n!}{2^{n}}\right)
\]


Во втором случае результат тоже равен $\underline{O}\left(\frac{\left(n-1\right)!}{2^{n}}\right)$.

\rule[0.5ex]{0.97\columnwidth}{1pt}

Тут вся сумма эквивалентна последнему слагаемому $n!/\left(2^{n}\right)$.
Начиная с какого-то момента, каждое следующее больше предыдущего хотя
бы в два раза, сумма слагаемых до этого момента - константа. Сумму
кроме последнего можно оценить суммой бесконечной геом. прогрессии
с начальным членом $\frac{\left(n-1\right)!}{2^{n-1}}$ и знаменателем
$\nicefrac{1}{2}$, плюс константа, что равно $\underline{O}\left(\frac{\left(n-1\right)!}{2^{n-1}}\right)$,
а значит $\bar{o}\left(\frac{n!}{2^{n}}\right)$. Теперь избавимся
от факториала с помощью формулы Стирлинга
\[
\frac{n!}{2\cdot n}\sim\sqrt{2\cdot\pi\cdot n}\cdot\left(\frac{n}{2e}\right)^{n}
\]


\rule[0.5ex]{0.97\columnwidth}{1pt}

В рассматриваемой сумме вынесем последнее слагаемое в виде множителя
из суммы, тогда получим: 
\[
\sum_{k=1}^{n}\frac{k!}{2^{k}}=\frac{n!}{2^{n}}\cdot\sum_{k=1}^{n}\frac{2^{n-k}}{\nicefrac{n!}{k!}}
\]


Записывая получившуюся сумму начиная с последнего слагаемого и далее
факторизуя, можно расписать в виде вложенных скобок: 
\[
\sum_{k=n}^{1}\frac{k!}{2^{k}}=1+\frac{2}{n}+\frac{4}{n\cdot\left(n-1\right)}+...+\frac{2^{n-1}}{n!}
\]


\[
\sum_{k=n}^{1}\frac{k!}{2^{k}}=1+\frac{2}{n}+\frac{2\cdot2}{n\cdot\left(n-1\right)}+...+\frac{2^{n-1}}{n!}=1+\frac{2}{n}\cdot\left(1+\frac{2}{n-1}\cdot\left(1+...\right)\right)
\]


Обозначая эту сумму через $S_{n}$, из вложенных скобок видно, что
эта сумма удовлетворяет рекурсии: $S_{n}=1+\frac{2}{n}\cdot S_{n-1}$.

Рассматривая это равенство при $n\rightarrow\infty$, получим: $S_{n}=1+\bar{o}\left(S_{n}\right)$,
это означает, что асимптотически $S_{n}\sim1$.

Значит искомая сумма имеет асимптотику 
\[
\sum_{k=1}^{n}\frac{k!}{2^{k}}=\frac{n!}{2^{n}}\cdot\sum_{k=1}^{n}\frac{2^{n-k}}{\nicefrac{n!}{k!}}=\frac{n!}{2^{n}}\cdot S_{n}\sim\frac{n!}{2^{n}}\sim\sqrt{2\cdot\pi\cdot n}\cdot\left(\frac{n}{2e}\right)^{n}
\]


Последняя эквивалентность записана на основе формулы Стирлинга, так
как в условии просили обойтись без факториалов. Для подтверждения
можно построить график с первым и вторым (добавлен множитель $\left(1+\nicefrac{2}{n}\right)$)
приближениями.

\includegraphics{3\lyxdot 3_step_8}
\end{document}
