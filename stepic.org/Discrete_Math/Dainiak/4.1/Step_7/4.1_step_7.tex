%% LyX 2.1.2 created this file.  For more info, see http://www.lyx.org/.
%% Do not edit unless you really know what you are doing.
\documentclass[russian]{article}
\usepackage[T2A,T1]{fontenc}
\usepackage[utf8]{inputenc}
\usepackage{geometry}
\geometry{verbose,tmargin=0.75cm,bmargin=2cm,lmargin=1.5cm,rmargin=1cm}
\usepackage{amssymb}

\makeatletter

%%%%%%%%%%%%%%%%%%%%%%%%%%%%%% LyX specific LaTeX commands.
\DeclareRobustCommand{\cyrtext}{%
  \fontencoding{T2A}\selectfont\def\encodingdefault{T2A}}
\DeclareRobustCommand{\textcyr}[1]{\leavevmode{\cyrtext #1}}
\AtBeginDocument{\DeclareFontEncoding{T2A}{}{}}


\makeatother

\usepackage{babel}
\begin{document}

\title{Теорема Рамсея, числа Рамсея | Шаг 3 \\
\rule[0.5ex]{1\columnwidth}{1pt}}

\maketitle
Давайте вспомним вариант формулировки теоремы Рамсея в терминах раскрасок
и рассмотрим такое обобщение теоремы Рамсея на три цвета. Для любых
натуральных чисел $r,s,t$ существует $n$, такое, что, как бы мы
ни раскрасили рёбра полного графа .$K_{n}$. в красный, зелёный и
синий цвета, найдётся хотя бы один из трёх подграфов: 
\begin{itemize}
\item полный подграф на $r$ вершинах, всё рёбра которого красные,
\item полный подграф на $s$ вершинах, всё рёбра которого зелёные,
\item полный подграф на $t$ вершинах, всё рёбра которого синие.
\end{itemize}
Обозначим через $R\left(r,s,t\right)$ минимальное $n$ , которое
гарантирует нам наличие таких подграфов, и будем называть такое число
трёхцветным числом Рамсея.

Примем без доказательство справедливость упомянутой теоремы. Докажите,
что трёхцветное число Рамсея удовлетворяет неравенству
\[
R\left(r,s,t\right)\leqslant R\left(r-1,s,t\right)+R\left(r,s-1,t\right)+R\left(r,s,t-1\right)-1.
\]


\rule[0.5ex]{0.27\columnwidth}{1pt}

Комментарий от преподавателя:

Доказательство почти слово в слово повторяет доказательство для обычных
чисел Рамсея, с той лишь разницей, что теперь мы должны выделить вершину
$v$ и рассмотреть три множества: те вершины, которые соединены с
$v$ красными рёбрами, синими и зелёными.\\
\rule[0.5ex]{1\columnwidth}{1pt}

Докажем, что трехцветное число Рамсея удовлетворяет неравенству

${R\left(r,s,t\right)\leqslant R\left(r-1,s,t\right)+R\left(r,s-1,t\right)+R\left(r,s,t-1\right)-1}.$

Для этого рассмотрим полный граф $K_{n}$ с числом вершин $n=R\left(r-1,s,t\right)+R\left(r,s-1,t\right)+R\left(r,s,t-1\right)-1$
и докажем, что как бы мы ни раскрасили его ребра в красный, зеленый
и синий цвета, в нем обязательно найдется хотя бы один из трех подграфов:
полный подграф на $r$ вершинах, все ребра которого красные, либо
полный подграф на $s$ вершинах, все ребра которого зеленые, либо
полный подграф на $t$ вершинах, все ребра которого синие.

Выделим в графе $K_{n}$ произвольную вершину $v$, а остальные вершины
разобьем на три не пересекающихся множества: вершины, которые соединены
с $v$ красным ребром, поместим в множество $N_{r}$, вершины, которые
соединены с $v$ зеленым ребром, поместим в множество $N_{g}$ и наконец
вершины, которые соединены с $v$ синим ребром, поместим в множество
$N_{b}$. Так как граф полный, то никаких <<не учтенных>> вершин
в нем не осталось, каждая его вершина (кроме $v$) принадлежит одному
из указанных множеств, то есть $\left|N_{r}\right|+\left|N_{g}\right|+\left|N_{b}\right|+1=n$.
Также важно заметить, что вершины каждого из множеств образуют полные
подграфы (клики), просто потому, что в полном графе все вершины по
определению соединены между собой.

\sloppyДокажем, что выполняется хотя бы одно из неравенств: $\left|N_{r}\right|\geqslant R\left(r-1,s,t\right)$
или $\left|N_{g}\right|\geqslant R\left(r,s-1,t\right)$ или ${\left|N_{b}\right|\geqslant R\left(r,s,t-1\right)}$.\fussyПредположим,
что это не так. Тогда должны \textit{одновременно} выполняться три
неравенства: ${\left|N_{r}\right|\geqslant R\left(r-1,s,t\right)}$
и ${\left|N_{g}\right|\geqslant R\left(r,s-1,t\right)}$ и ${\left|N_{b}\right|\geqslant R\left(r,s,t-1\right)}$

.\\
\rule[0.5ex]{1\columnwidth}{1pt}

.\\
\rule[0.5ex]{1\columnwidth}{1pt}

.

\ \\
\rule[0.5ex]{1\columnwidth}{1pt}
\end{document}
